\documentclass{article}
\usepackage{amsmath}
\usepackage{titlesec}
\usepackage[mathletters]{ucs}
\usepackage[utf8x]{inputenc}
\usepackage[margin=1.5in]{geometry}
\usepackage{enumerate}
\newtheorem{theorem}{Theorem}
\usepackage[dvipsnames]{xcolor}
\usepackage{pgfplots}
\pgfplotsset{compat=1.18}
\setlength{\parindent}{0cm}
\usepackage{graphics}
\usepackage{graphicx} % Required for including images
\usepackage{subcaption}
\usepackage{bigintcalc}
\usepackage{pythonhighlight} %for pythonkode \begin{python}   \end{python}
\usepackage{appendix}
\usepackage{arydshln}
\usepackage{physics}
\usepackage{booktabs} 
\usepackage{adjustbox}
\usepackage{mdframed}
\usepackage{relsize}
\usepackage{physics}
\usepackage[thinc]{esdiff}
\usepackage{esint}  %for lukket-linje-integral
\usepackage{xfrac} %for sfrac
\usepackage[colorlinks=true]{hyperref} %for linker, må ha med hypersetup
\usepackage[noabbrev, nameinlink]{cleveref} % to be loaded after hyperref
\usepackage{amssymb} %\mathbb{R} for reelle tall, \mathcal{B} for "matte"-font
\usepackage{listings} %for kode/lstlisting
\usepackage{verbatim}
\usepackage{graphicx,wrapfig,lipsum,caption} %for wrapping av bilder
\usepackage{mathtools} %for \abs{x}
\usepackage[english]{babel}
\usepackage{cancel}
\usepackage{mhchem} % for atom notasjon
\definecolor{codegreen}{rgb}{0,0.6,0}
\definecolor{codegray}{rgb}{0.5,0.5,0.5}
\definecolor{codepurple}{rgb}{0.58,0,0.82}
\definecolor{backcolour}{rgb}{0.95,0.95,0.92}
\lstdefinestyle{mystyle}{
    backgroundcolor=\color{backcolour},   
    commentstyle=\color{codegreen},
    keywordstyle=\color{magenta},
    numberstyle=\tiny\color{codegray},
    stringstyle=\color{codepurple},
    basicstyle=\ttfamily\footnotesize,
    breakatwhitespace=false,         
    breaklines=true,                 
    captionpos=b,                    
    keepspaces=true,                 
    numbers=left,                    
    numbersep=5pt,                  
    showspaces=false,                
    showstringspaces=false,
    showtabs=false,                  
    tabsize=2
}

\lstset{style=mystyle}
\author{Oskar Idland}
\title{Midterm}
\date{}
\begin{document}
\maketitle
%\tableofcontents
\newpage

\section{Cross Sections}
\subsection*{a)}


\subsection*{b)}


\section{Nuclear Properties and Force}
\subsection*{a)}
\begin{itemize}
    \item \textbf{Rutherford scattering experiment} taught us that the atom is mostly empty space, with a small, dense nucleus carrying the positive charge. This is because the alpha particles, which are positively charged, are deflected by the nucleus, which is also positively charged. Many alpha particles pass through the gold foil without being deflected, which means that the nucleus is very small compared to the atom.
    \item \textbf{Chadwick's scattering experiment} showed that the nucleus also contains neutral particles, named neutrons. This was discovered by bombarding beryllium with alpha particles, which resulted in the emission of neutral radiation.
    \item \textbf{Electron scattering experiments} showed that the nucleus is not a point particle but has a radius which increases with the atomic number $A$ (or mass) which it's charge is distributed over.
    \item \textbf{Hadron scattering experiments} taught us the mass distribution of the nucleus is proportional to the nuclear force and not the Coulomb force.
\end{itemize}


\subsection*{b)}
\textbf{Yes}, nuclear radius is a definitive property of the nucleus. Still, it is not entirely accurate, as the nucleus does not have a well-defined surface. This is shown by electron scattering as the diffraction pattern is not exactly that of a circular disk.


\subsection*{c)}
\begin{enumerate}
    \item \textbf{Charge Independence:} The nuclear force is nearly independent of the charge of the nucleons. We know this from experiments on excited states of \textit{mirror nuclei} like $\ce{_{6}^{11}\text{C}_{5}}$ and $\ce{_{5}^{11}\text{B}_{6}}$. 
    \item \textbf{Attractive and Repulsive:} The nuclear force is strongly attractive distances shorter than 1-2 fm and negligible at greater distances. The nuclear force is repelling at distances shorter than 1 fm. We know the nuclear force is attractive at those smaller distances because of Rutherford scattering experiments. Here we see that the alpha particles needs a certain amount of energy to overcome the Coulomb force and get close enough to the nucleus to be attracted. The linear dependence of the binding energy per nucleon also shows that not all nucleons are attracted by each other at the same time, only the nearest neighbors. 
    \item \textbf{Electrons:} Electrons are not affected by the nuclear force (hence the name)
\end{enumerate}


\section{Nuclear Binding Energy}
\subsection*{a)}
The semi-empirical binding energy formula shows how the binding energy per nucleus depends on the number of electrons and protons as seen in \cref{eq: SEBE}.  When the atoms are small, the nucleons are all close together and therefore the protons are affected by both the nuclear force, and the Coulomb force. When the size of the atom increases, we need more neutrons to keep the nucleus stable, as the extra neutrons provide more attractive nuclear force to counteract the repulsive Coulomb force, and move the protons further apart which again reduces the Coulomb force. This keeps the nucleus stable, but makes the $N / Z$ ration less than one for larger nuclei.   
\begin{equation}\label{eq: SEBE}
B(A,Z) = a_{V}A - a_{S}A^{2/3} - a_{C}\frac{Z(Z-1)}{A^{1/3}} - a_{A}\frac{(A-2Z)^{2}}{A} + \delta(A,Z)
\end{equation}

\subsection*{b)}
There are three components to the semi-empirical binding energy formula as seen in \cref{eq: SEBE}. 
\begin{enumerate}
    \item $\mathbf{a_vA}$: \textbf{Volume term}. The binding energy is proportional to the volume of the nucleus approximated to a sphere $\left(V = 4πR^3/3\right)$. This dominates the binding energy for large nuclei. 
    \begin{equation}
    a_v ≈ 15.8 \text{ MeV}
    \end{equation}
    The linear dependence of the binding energy on the number of nucleons tells us that the strong force is short range as each nucleon only interacts with its nearest neighbors. 
    \item $\mathbf{a_sA^{2 / 3}}$: \textbf{Surface term}. The volume term is not quite accurate as the nucleons on the surface have fewer neighbors. This term corrects for that. The binding energy is proportional to $πR^2$
    \begin{equation}
    a_s ≈ 16.8 \text{ MeV}
    \end{equation} 
    \item $\mathbf{a_cZ(Z-1)A^{-1 / 3}}$: \textbf{Coulomb term}. The binding energy is reduced by the repulsion between the protons. It is therefore detracted. The Coulomb force is long range and is therefore proportional to $Z(Z-1)$ as all protons interact. 
    \begin{equation}
    a_c ≈ 0.72 \text{ MeV}
    \end{equation}
\end{enumerate}

\subsection*{c)}


\subsection*{d)}



\section{Nuclear Shell Model}
\subsection*{a)}


\subsection*{b)}


\subsection*{c)}


\subsection*{d)}


\subsection*{e)}



\section{Relativistic B-mesoon pairs at SuperKEKB}
\subsection*{a)}


\subsection*{b)}


\subsection*{c)}


\subsection*{d)}




\section{Neutrino Oscillations}
\subsection*{a)}


\subsection*{b)}



\section{Quantum Numbers}

\subsection*{a)}


\subsection*{b)}


\subsection*{c)}
Only $π^{0}$ is an eigenstate of C-parity. This is because if you flip the charge, you still have the same particle (aka. quantum state). Flipping the charge of $π^{+}$ gives $π^{-}$, and flipping the charge of $π^{-}$ gives $π^{+}$, which are different particles and therefore not an eigenstate. 


\section{Allowed, Suppressed and Forbidden Processes}
\subsection*{a)}


\subsection*{b)}


\subsection*{c)}


\subsection*{d)}




\end{document}