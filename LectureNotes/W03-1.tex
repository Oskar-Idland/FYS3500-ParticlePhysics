\section{History}
\begin{itemize}
    \item 1896: Henri Becquerel discovered radioactivity
    \item 1898: Marie and Pierre Curie discovered radium and polonium
    \item 1903: Alphas charge to mass ratio
    \item 1909: Alphas are helium nuclei
    \item 1911: Rutherford discovers the nucleus
    \item 1913: Bohr model of the atom
    \item 1917: Rutherford discovers the proton
    \item 1930: Neutrinos were postulated 
    \item 1932: Chadwick discovers the neutron by shooting alpha particles at beryllium. 
    \item 1938: Discovery of nuclear fission
    \item 1956: Neutrinos were detected 
\end{itemize}

\subsection{Proton Discovery: The Rutherford Scattering Experiment}

Thomson's model of the atom was a positive sphere with electrons embedded in it. Rutherford wanted to test this model by shooting alpha particles at a thin gold foil surrounded by a detector foil. The alpha particles were shot from a radioactive source and when the alpha particles exited, they hit the foil and emitted light. 
\subsubsection{Conclusion}
\begin{itemize}
    \item Most alpha particles went straight through the foil. This implies the atom is mostly empty space.
    \item Some alpha particles were deflected by a small angle. This implies the positive charge is concentrated in a small volume.
    \item Sometimes the particles travel backwards. This implies the positive center has most of the mass of the atom.
\end{itemize}

\subsection{Discovery of the Neutron}
\begin{itemize}
    \item Shooting alpha particles on beryllium which is much lighter than gold. This 
\end{itemize}

\section{Nucleus}
\begin{itemize}
    \item Very dense. Carries all the mass. $2.7 ⋅ 10^{14}$ times denser than water.
    \item The atom is mostly empty space. If the nucleus was the size of a coin, the atom would be 2-3 km in radius. 
\end{itemize}
\subsection{Notation}
\begin{itemize}
    \item \textbf{Notation}: $_{Z}^{A}X_{N}$ 
    \item Isoto\textbf{p}e: Same \textbf{proton} number $Z$
    \item Isoto\textbf{n}e: Same \textbf{neutron} number $N$
    \item Isob\textbf{a}r: Same \textbf{atomic} mass number $A = Z + N$
\end{itemize}

\subsection{Nuclides}
\begin{itemize}
    \item 92 stable elements
    \item 280 stable isotopes
    \item 3000 unstable isotopes
    \item 6000 more predicted to exist
\end{itemize}
\subsubsection{Stable Numbers}
\begin{equation}
N = 2, 8, 20, 28, 50, 82, 126
\end{equation}
\begin{equation}
Z = 2, 8, 20, 28, 50, 82, \ldots     
\end{equation}
