\section{Neutrino Mixing Continued}
\begin{itemize}
    \item \textbf{Sterile Neutrinos}: A hypothetical fourth neutrino generation that does not interact with the weak force. It will therefore not be detected by the standard model detectors, but a gap in the number of neutrons after a reaction would point to its existence.
\end{itemize}

\subsection{Neutrino Mixing in Practice}
\begin{itemize}
    \item The different neutrinos interact with the electron through the $Z^{0}$-boson all the same. 
    \item The electron neutrino can also interact with the electron, through the $W^{-}$-boson. This would change the charge of each particle. 
\end{itemize}

\subsection{Quark Mixing}
\begin{itemize}
    \item Three generations of quarks: 
    \begin{enumerate}
        \item Up and Down $\displaystyle \begin{pmatrix*}[r]
         u \\
         d \\
        \end{pmatrix*}$
        \item Charm and Strange $\displaystyle \begin{pmatrix*}[r]
         c \\
         s \\
        \end{pmatrix*}$
        \item Top and Bottom $\displaystyle \begin{pmatrix*}[r]
         t \\
         b \\
        \end{pmatrix*}$
    \end{enumerate}
    \item The mass increases for each generation.
    \item Top-type quarks have a charge of $+\frac{2}{3}$, while bottom-type quarks have a charge of $-\frac{1}{3}$.
    \item All quarks are spin $\frac{1}{2}$ particles.
    \item No free quarks has been observer. We look at hadrons for information about quarks.
    \item High energy collisions of particle and anti-particles creates multiple jets of particles. This shows that there are multiple quarks in the hadrons.
    \item The top quark has such a short lifetime that it decays before it can form a hadron. It also has the greatest mass. 
    \item \textbf{Spectator Model}: A single quark involved in a hadron decay, assumes the other quarks are spectators and does not interact. 
    
    
\end{itemize}
\subsubsection{Quark Number \& Flavour Conservation}
\begin{itemize}
    \item Quark number for each flavour is conserved for electromagnetic and strong interactions. 
    \begin{equation}
      N_f ≡ N(f) - N(\bar{f}) \quad , \quad  f ∈ \left\{u, d, s, c, b, t\right\}
    \end{equation} 
    \item The flavour of the quarks are allowed to change through the weak interaction. This means the following interaction would be allowed:
    \begin{equation}
      c \overset{\underset{W^{+}}{}}{→} s u \bar{d} 
    \end{equation} 
    \item Baryon number is conserved in all interactions. The baryon number is defined as follows:
    \begin{equation}
      B ≡ N_q / 3 = \left(N(q) - N(\bar{q})\right) / 3
    \end{equation}
\end{itemize}

\subsection{Hadrons}
The most common hadrons are:
\begin{itemize}
    \item Baryons made up of three quarks. 
    \item Anti-baryons made up of three anti-quarks.
    \item Mesons made up of a quark and an anti-quark.
\end{itemize}

The strong force does not care about the flavour. The forces between the quarks are the same for all quarks. 

\subsection{Isospin}
\begin{itemize}
    \item The symmetry between the up and down quarks, with respect to the strong interaction. 
    \item Has nothing to do with regular spin. 
    \item The up and down sates are flavour states of the quarks. 
    \item The same can be used for protons and neutrons as two states of the nucleon. 
    \item The symmetry is just an approximation, as the masses and charge of the up and down quarks are not the same.
\end{itemize}

\subsubsection{Isospin Doublet}
\begin{itemize}
    \item The "light quark" and anti-light quark are noted as follows:
    \begin{equation}
        \begin{pmatrix*}[r]
        u \\
        d \\
        \end{pmatrix*} \quad , \quad \begin{pmatrix*}[r]
        \bar{u} \\
        -\bar{d} \\
        \end{pmatrix*}
    \end{equation} 
    \item In this framework, we can describe the nucleon as a state of the isospin doublet.
    \begin{equation}
        \begin{pmatrix*}[r]
        p \\
        n \\
        \end{pmatrix*} = 
        \begin{pmatrix*}[r]
         udu \\
         udd \\
        \end{pmatrix*} = 
        (ud) 
        \begin{pmatrix*}[r]
         u \\
         d \\
        \end{pmatrix*}
    \end{equation} 
    \item Example of the "Kaon":
    \begin{equation}
        \begin{pmatrix*}[r]
        K^{+} \\
        K^{0} \\
        \end{pmatrix*} = 
        \begin{pmatrix*}[r]
         u \bar{s} \\
         d \bar{s} \\
        \end{pmatrix*} =  
        \begin{pmatrix*}[r]
         u \\
         d \\
        \end{pmatrix*}
        (\bar{s})
    \end{equation} 
    \item Perfect isospin symmetry would mean that the mass of the proton and neutron would be the, just different charge. 
\end{itemize}

\subsection{Isospin Mathematics}
\begin{itemize}
    \item Same as with spin and angular momentum. 
    \begin{itemize}
        \item If $I = I_3$, we get $I_3 = -I, -I + 1, \ldots  , I - 1, I$
        \item If $I = 2 / 3$ we get $I_3 = -3 / 2, -1 / 2, 1 / 2, 3 / 2$
    \end{itemize}
    \item Addition works the same as for spin and angular momentum. 
    \begin{itemize}
        \item $I^{a} + I^{b} = \left|I^{a} - I^{b}\right|, \left|I^{a} - I^{b}\right| + 1, \ldots  ,I^{a} + I^{b}-1, I^{a} + I^{b}$
        \item If we have $I^{a} = I^{b} = 1$, we get: $1 + 1 = 0, 1, 2$
    \end{itemize} 
    \item For isospin conserving strong interactions we have the same sum-rule for the third $z$-component. 
    \begin{equation}
      I_3 = ∑_{i}^{} I_3^{i} \text{ where } I_3 (u, \overline{b}) = 1 / 2 \text{ and } I_3 (\bar{u}, d) = -1 / 2
    \end{equation}
\end{itemize}

\subsubsection{Isospin of Deuteron}
A deuteron is a proton and neutron. We need to use the sum rule, and third component of isospin to find the isospin of the deuteron.
\paragraph{Proton:}
\begin{equation}
  \ket{p, I, I_3} = \ket{\frac{1}{2}, \frac{1}{2}}
\end{equation}
\paragraph{Neutron:}
\begin{equation}
    \ket{n, I, I_3} = \ket{\frac{1}{2}, -\frac{1}{2}}
\end{equation}

\paragraph{Deuteron:}
\begin{equation}
  I_3(d) = I_3(p) + I_3(n) = \frac{1}{2} - \frac{1}{2} = 0
\end{equation}
We do not have enough information to find $I$. We do know that no other charge state with approximately the same mass, meaning the deuteron must be in a iso-singlet state of:
\begin{equation}
  \ket{I, I_3} = \ket{0, 0}
\end{equation}

\subsubsection{Isospin of a Pion and nucleon}
Given a final state with a pion $I = 1$ and a nucleon $I = 1 / 2$, what are the possible initial isospin states?

\paragraph{Solution 1:}
\begin{equation}
  \ket{I, I_3}  = \ket{\frac{1}{2}, ±\frac{1}{2}}
\end{equation}

\paragraph{Solution 2:}
\begin{equation}
\ket{I, I_3}  = \left\{\ket{\frac{3}{2}, ±\frac{3}{2}} , \ket{\frac{3}{2}, \pm \frac{1}{2}}\right\}
\end{equation}

\section{Spectroscopic Notation}
\begin{equation}
^{2S + 1}L_{J}
\end{equation}
where $S$ is the total spin of the continuans, $L$ is the total orbital angular momentum of the continuans. $J$ is the sum of the two, and can take many values. 
\begin{equation}    
  J = \vec{L} + \vec{S} → J = \left|L - S\right|, \left|L - S\right| + 1, \ldots  , \left|L + S - 1\right|,  \left|L + S\right|
\end{equation}

$L$ can take the following values:
\begin{table}[h!]
\begin{tabular}{ |c|c| }
\hline
L &Symbol \\ 
\hline
0 &S \\ 
1 &P \\ 
2 &D \\ 
3 &F \\ 
4 &G \\ 
\hline
\end{tabular}
\end{table}
$^{1}S_0$ means $J = S = L = 0$. 
