\subsection{Cross-Section}
\begin{itemize}
    \item We measure the rate $W$ of a process happening. 
    \item The number of particles hitting a target per unit area is called flux $J = n_{b} v_i$, where $n_{b}$ is the number of particles per unit volume and $v_i$ is the velocity of the beam. 
    \item The area per target particle is called the cross section $σ$. This represents the area in which the particles interact in the relevant process.
    \item The rate of particles hitting the target is $W = JNσ$. where $N$ is the number of particles. 
    \item The rate $W$ is proportional to the luminosity $L$ by $W = L(t)σ$. 
    \item The cross section $σ$ is invariant under Lorentz transformations in the direction of the beam, meaning it is independent of the reference frame. 
    \item Colliding two beams beams of $N_1$ number of particles and $N_2$ number of particles, we get a luminosity $L(t) = f N_1 N_2 / A$, where $f$ is the frequency of the beam (how often they collide) and $A$ is the area of the beam.
\end{itemize}

\subsubsection{Differential Cross-Section}
Measured time-integrated rate as a function of some observable. This could be momentum: 
\begin{equation}
    σ = ∫_{p_{\text{min}}}^{p_{\text{max}}} \frac{\mathrm{d}σ(p)}{\mathrm{d}p} \ \mathrm{d}p
\end{equation}

\subsubsection{Measured Cross-Section}
\begin{enumerate}
    \item Define the process of interest. 
    \item Count the number of times the process happens.
    \item Remove the background noise as measurement is imperfect. 
    \item Find the time-integrated luminosity $\mathcal{L} ≡ ∫  L(t) \mathrm{d}t$. 
    \item For the total cross-section, we have $N = σ \mathcal{L}ϵ + b$ or $\displaystyle σ = \frac{N - b}{\mathcal{L}ϵ}$, where $ϵ$ is the efficiency of the detector and $b$ is the background noise.
\end{enumerate}

\subsubsection{Integrated Luminosity}
\begin{itemize}
    \item One can either precisely control the details of the beam and target to determine $\mathcal{L}$.
    \item One can also used well-known processes to determine $\mathcal{L} = N_1 / σ_1$, if this event has a known cross-section. After that, one can correct for noise and efficiency, to then find the cross-section of the process of interest by $σ_2 = \frac{N_2}{\mathcal{L}} = \frac{N_2}{N_1} σ_1$
    \item Barn: Unit of cross-section, $1 \text{b} = 10^{-24} \text{cm}^2$ or $1 \text{b} = 10^{-28} \text{m}^2$. Integrated luminosity is often given in $\text{fb}^{-1}$ or $\text{pb}^{-1}$, where f $(10^{-15})$ and p $(10^{-12})$ stands for femto and pico, respectively. $1\text{pb} ≈ 10^{-40}\text{m}^2 = (10^{-5}\text{fm})^2$. In comparison the proton has a radius of about $0.84\text{fm}$.
\end{itemize}


\subsubsection{Single Beam Particle and Target}
\begin{itemize}
    \item Again we use that the rate of observing some scattering process $r$ is $W_r = JNσr$. 
    \item In the case of a single target particle we have $N = 1$. 
    \item The flux $J = n_{b} v_i$ where $n_{b}$ is the number of beam particles per unit volume $V$ and $v_i$ is the velocity of the beam. 
    \item The differential rate of observing the process $r$ in a solid angle at direction $(θ,ϕ)$, is therefore $ \displaystyle \mathrm{d}W_r = v_\frac{i}{V} \frac{\mathrm{d}σ_r(θ,ϕ)}{\mathrm{d}Ω}dΩ$
\end{itemize}

\subsubsection{Born Approximation}
\begin{itemize}
    \item Assuming the particle has a wavefunction traveling a large distance around the accelerator, we can describe it as a plane wave $ψ_i$. 
    \item When disturbed by a potential $V(r)$, we can calculate the rate of scattering by the Born approximation. This is also known as "Fermi's Golden Rule". If we define the probability amplitude $\mathcal{M}(\vec{q})$, we get from \cref{eq: fermis_golden_rule} that the rate of scattering is given by \cref{eq: fermis_golden_rule_2}. Here $\vec{q}$ is the momentum transferred. 
    \begin{equation}\label{eq: fermis_golden_rule}
      \mathrm{d}W_r = \frac{2π}{ℏ} \left|∫ ψ_f^{*}V(\vec{r})ψ_i \mathrm{d}^3 \vec{r}\right| ρ(E_f) 
    \end{equation}
    \begin{equation}\label{eq: fermis_golden_rule_2}
      \mathrm{d}W_r = \frac{2π}{ℏ} \left|M(\vec{q})\right|^2 ρ(E_f), \quad M ≡ ∫ V(\vec{r})\exp \left(i \vec{q} ⋅ \vec{r} / ℏ\right)d^3 \vec{r}
    \end{equation}
    \item Multiplying by $ℏ^2c^2$ which we know to be about $(197 \text{ MeV F})^2$ we cancel out the energy and get $σ$ in units of $\text{F}^2$ aka barn. 
\end{itemize}

\subsubsection{Natural Units and Cross Section}
\begin{itemize}
    \item Cross-sections $σ$ has unit of area, barn (b) of $10^{-28}$ m$^2$. 
    \item In natural units this is proportional to $1 / E^2$ with units $1 / $MeV$^2$
\end{itemize}

\subsection{Density of States (Energy States)}
The number of states in a range of momenta $d^3 \vec{p}$. This is derived from a wavefunction in a box of volume $V = L^3$ with walls of infinite potential. The coordinate system is places in one of the corners such that the walls are at $x = L$, $y = L$ and $z = L$. From this we get the density of states in momentum space $d^3 \vec{p}$ as seen in \cref{eq: density_of_states}.
\begin{equation}\label{eq: density_of_states}
 ρ(E) \mathrm{d}q = \frac{L^3}{(2πℏ)^3} d^3 \vec{p} \quad , \quad  → \mathrm{d}^3 \vec{p} = q^2 \ \mathrm{d}q \ \mathrm{d}ϕ \ \mathrm{d}(\cos ϕ) ≡ q^2 \mathrm{d}q \mathrm{d}Ω
\end{equation}
\cref{eq: density_of_states} can be used further where \cref{eq: density_of_states_2} is defined. Now we need to find $\frac{\mathrm{d}q}{\mathrm{d}E}$
\begin{equation}\label{eq: density_of_states_2}
    ρ(q)\mathrm{d}q ≡ ρ(E)\mathrm{d}E → ρ(E) = ρ(q) \frac{\mathrm{d}q}{\mathrm{d}E} = \frac{L^3}{(2πℏ)^3} q^2 \frac{\mathrm{d}q}{\mathrm{d}E} \mathrm{d}Ω
\end{equation}
\paragraph{Non-Relativistic Case:}
\begin{itemize}
    \item $E = q^2 / 2m$
    \item $\mathrm{d}E = 2q \mathrm{d}q / 2m = q \mathrm{d}q /m$
    \item \textbf{Conclusions:} $\frac{\mathrm{d}q}{\mathrm{d}E} = m / q = 1 / v$
\end{itemize}

\paragraph{Relativistic Case:}
\begin{itemize}
    \item $E^2 = m^2c^{4} + q^2 c^2$
    \item $2E \mathrm{d}E = 2c^2 q \mathrm{d}q$
    \item \textbf{Conclusions:} $\frac{\mathrm{d}q}{\mathrm{d}E} = E / qc^2 = mγc^2 / mγβc^3 =  \frac{1}{ βc} = 1 / v$
    The result is the same!
\end{itemize}

\subsubsection{Conclusion}
\begin{equation}
\frac{\mathrm{d}σ_r(θ,ϕ)}{\mathrm{d}Ω} = \frac{1}{(2π^2)} \frac{1}{ℏ^{4}} \frac{q_f^2}{v_f v_i} \left|M(\vec{q})\right|^2
\end{equation}

\subsection{Theoretical Cross-Section for $ab → cd$}
\begin{itemize}
    \item The differential cross-section for a process $ab → cd$ is given by \cref{eq: differential_cross_section}, where $g_i$ and $g_f$ are number of spin-states for the initial and final states particles, respectively. $g_f = (2S_{c} + 1)(2S_{d} + 1)$ and $g_i = (2S_{a} + 1)(2S_{b} + 1)$. $k$ represent all possible spin states. 
    \begin{equation}\label{eq: differential_cross_section}
    \frac{\mathrm{d}σ}{\mathrm{d}Ω} = \frac{g_f}{4π^2ℏ^{4}} \frac{q_f^2}{v_i v_f} \overline{|M(\vec{q})|^2} \quad , \quad  \overline{|M(\vec{q})|^2} = \frac{1}{g_i g_f} ∑_{k}^{} \left|M_k\right|^2
    \end{equation}
    \item In general, the particle beams are unpolarized and we therefore do not measure the spin of the final state particles. 
    \item We average over the initial spin states, and sum over the final ones. 
\end{itemize}

\section{Particle Phenomenology}
\subsection{Leptons}
\begin{itemize}
    \item 3 generations increasing in mass. 
    \item Neutrinos has only weak interactions due to lack of charge. This makes it has to detect and is a hint of a weak interaction. 
    \item Except a few cases, the generations of leptons are conserved in interactions.
    \item The weakness of the weak force comes from the massive mediating particles $W^{\pm}$ and $Z^{0}$. 
    \item The flavour of the lepton does not affect the strength of the interaction. 
\end{itemize}