
We only take into account the last nucleon as the total parity of all paired nucleons is $+1$. No matter if the pair has both parity $+1$ or $-1$, the total parity is $+1$, as they are multiplied. 
\begin{equation}
  π = ∏_{i}^{A}  π_i = (-1)^{l_i}
\end{equation}

The Shell Model can predict many things, such as:
\begin{itemize}
    \item Ground state spin-parity 
    \item Excited states 
    \item Magnetic moments
    \item Electric moments 
\end{itemize}

\subsubsection{Ground State Spin-Parity of Odd-Odd Nuclei}
In the case of $\ce{_{49}^{110}\text{In}_{61}}$, we cannot get a single answer. The answer can be in the range of $\left|j_p - j_n\right| ≤ I ≤ j_p + j_n$. The parity is then $π = (-1)^{l_p} (-1)^{l_p}$. In this case, we have the proton at $1g_{9/2}$ with $j_p = 9/2$ and the neutron at $2d_{5/2}$ $j_n = 5/2$. This gives the possible values of $I ∈ \left\{2, 3, 4, 5, 6, 7\right\}$, with a parity of $π = (+)(+)$

\subsubsection{Ground State Spin-Parity of Even-Even Nuclei}
This is the easiest case. It is always $(+)$ as each nucleon has a partner with the same parity, and will therefore always give a positive parity.