\documentclass{article}
\usepackage{amsmath}
\usepackage{titlesec}
% \usepackage[mathletters]{ucs}
\usepackage{mathtools} %for \abs{x}
\usepackage[warnings-off={mathtools-colon,mathtools-overbracket}]{unicode-math}
% \setmainfont{TeX Gyre Schola}
% \setmathfont{TeX Gyre Schola Math}
% \usepackage[utf8x]{inputenc}
\usepackage{fontenc}
\usepackage[margin=1.5in]{geometry}
\usepackage{enumerate}
\newtheorem{theorem}{Theorem}
\usepackage[dvipsnames]{xcolor}
\usepackage{pgfplots}
\pgfplotsset{compat=1.18}
\setlength{\parindent}{0cm}
\usepackage{graphics}
\usepackage{graphicx} % Required for including images
\usepackage{subcaption}
\usepackage{bigintcalc}
\usepackage{pythonhighlight} %for pythonkode \begin{python}   \end{python}
\usepackage{appendix}
\usepackage{arydshln}
\usepackage{physics}
\usepackage{booktabs} 
\usepackage{adjustbox}
\usepackage{mdframed}
\usepackage{relsize}
\usepackage{physics}
\usepackage[thinc]{esdiff}
\usepackage{esint}  %for lukket-linje-integral
\usepackage{xfrac} %for sfrac
\usepackage[colorlinks=true,linktoc=page]{hyperref} %for linker, må ha med hypersetup
\usepackage[noabbrev, nameinlink]{cleveref} % to be loaded after hyperref
% \usepackage{amssymb} %\mathbb{R} for reelle tall, \mathcal{B} for "matte"-font
\usepackage{listings} %for kode/lstlisting
\usepackage{verbatim}
\usepackage{graphicx,wrapfig,lipsum,caption} %for wrapping av bilder
\usepackage[english]{babel}
\usepackage{cancel}
% \usepackage{alphabeta}
\usepackage{mhchem} % for atom notasjon
% \definecolor{codegreen}{rgb}{0,0.6,0}
% \definecolor{codegray}{rgb}{0.5,0.5,0.5}
% \definecolor{codepurple}{rgb}{0.58,0,0.82}
% \definecolor{backcolour}{rgb}{0.95,0.95,0.92}
% \lstdefinestyle{mystyle}{
%     backgroundcolor=\color{backcolour},   
%     commentstyle=\color{codegreen},
%     keywordstyle=\color{magenta},
%     numberstyle=\tiny\color{codegray},
%     stringstyle=\color{codepurple},
%     basicstyle=\ttfamily\footnotesize,
%     breakatwhitespace=false,         
%     breaklines=true,                 
%     captionpos=b,                    
%     keepspaces=true,                 
%     numbers=left,                    
%     numbersep=5pt,                  
%     showspaces=false,                
%     showstringspaces=false,
%     showtabs=false,                  
%     tabsize=2
% }

% \lstset{style=mystyle}

\author{Oskar Idland}
\title{Problem Set 3}
\date{}
\begin{document}
% \tableofcontents
\maketitle
\newpage
\subsection*{Problem 1}
\subsection*{a)}
Fermions have intrinsic parity $+1$, and anti-fermions have intrinsic parity $-1$. The total parity is therefore:
\begin{equation}
  π = π_{u}π_{\bar{d}}(-1)^{l} = (-1)^{l+1}
\end{equation}

\subsection*{b)}
\begin{equation}
  π = π_f π_f (-1)^{l=1} = -1
\end{equation}

\subsection*{c)}
\begin{mdframed}
  \textbf{Verify that the spherical harmonic $Y_1^{1} = \sqrt{\frac{3}{8π}}\sin θ e^{iϕ}$ is an eigenfunction of parity with eigenvalue $P = -1$}
\end{mdframed}
The parity operator just flips the spatial coordinates. $\hat{P}(θ,ϕ) = (π-θ,π+ϕ)$
\begin{equation}
  \hat{P}Y_1^{1} = \sqrt{\frac{3}{8π}} \sin (π-θ) e^{i(ϕ + π)}
\end{equation}
Using the fact that $\sin θ = \sin (π-θ)$ and $e^{iπ} = -1$ we get:
\begin{equation}
  \hat{P}Y_1^{1} = - \sqrt{\frac{3}{8π}}\sin θ e^{iϕ} = - Y_1^{1}
\end{equation}


\section*{Problem 2}
\subsection*{a)}
\begin{itemize}
  \item The energy of the particle beam. At very high energies it might seem like the barrier does not exist. 
  \item The type of particle in the beam. If the particles interact very little (like neutrinos), their cross-section will be quite small. 
  \item The efficiency of the detector. 
  \item Density of particle stream and target. 
\end{itemize}

\end{document}