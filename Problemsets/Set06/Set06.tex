\documentclass{article}
\usepackage{amsmath}
\usepackage{titlesec}
% \usepackage[mathletters]{ucs}
\usepackage{mathtools} %for \abs{x}
\usepackage[warnings-off={mathtools-colon,mathtools-overbracket}]{unicode-math}
% \setmainfont{TeX Gyre Schola}
% \setmathfont{TeX Gyre Schola Math}
% \usepackage[utf8x]{inputenc}
\usepackage{fontenc}
\usepackage[margin=1.5in]{geometry}
\usepackage{enumerate}
\newtheorem{theorem}{Theorem}
\usepackage[dvipsnames]{xcolor}
\usepackage{pgfplots}
\pgfplotsset{compat=1.18}
\setlength{\parindent}{0cm}
\usepackage{graphics}
\usepackage{graphicx} % Required for including images
\usepackage{subcaption}
\usepackage{bigintcalc}
\usepackage{pythonhighlight} %for pythonkode \begin{python}   \end{python}
\usepackage{appendix}
\usepackage{arydshln}
\usepackage{physics}
\usepackage{booktabs} 
\usepackage{adjustbox}
\usepackage{mdframed}
\usepackage{relsize}
\usepackage{physics}
\usepackage[thinc]{esdiff}
\usepackage{esint}  %for lukket-linje-integral
\usepackage{xfrac} %for sfrac
\usepackage[colorlinks=true,linktoc=page]{hyperref} %for linker, må ha med hypersetup
\usepackage[noabbrev, nameinlink]{cleveref} % to be loaded after hyperref
% \usepackage{amssymb} %\mathbb{R} for reelle tall, \mathcal{B} for "matte"-font
\usepackage{listings} %for kode/lstlisting
\usepackage{verbatim}
\usepackage{graphicx,wrapfig,lipsum,caption} %for wrapping av bilder
\usepackage[english]{babel}
\usepackage{cancel}
% \usepackage{alphabeta}
\usepackage{mhchem} % for atom notasjon
% \definecolor{codegreen}{rgb}{0,0.6,0}
% \definecolor{codegray}{rgb}{0.5,0.5,0.5}
% \definecolor{codepurple}{rgb}{0.58,0,0.82}
% \definecolor{backcolour}{rgb}{0.95,0.95,0.92}
% \lstdefinestyle{mystyle}{
%     backgroundcolor=\color{backcolour},   
%     commentstyle=\color{codegreen},
%     keywordstyle=\color{magenta},
%     numberstyle=\tiny\color{codegray},
%     stringstyle=\color{codepurple},
%     basicstyle=\ttfamily\footnotesize,
%     breakatwhitespace=false,         
%     breaklines=true,                 
%     captionpos=b,                    
%     keepspaces=true,                 
%     numbers=left,                    
%     numbersep=5pt,                  
%     showspaces=false,                
%     showstringspaces=false,
%     showtabs=false,                  
%     tabsize=2
% }

% \lstset{style=mystyle}

\author{Oskar Idland}
\title{Problem Set 6}
\date{}
\makeatletter
\newcommand{\leqnomode}{\tagsleft@true\let\veqno\@@leqno}
\begin{document}
% \tableofcontents
\maketitle
\newpage
\section*{Problem 1 (3.16)}
\begin{mdframed}
Draw the lowest-order Feynman diagrams at the quark level for the following decays:
\begin{equation}\leqnomode
  \tag{a} \hspace{-9.5cm}  D^{-} →  K^{0} + π^{-}
\end{equation}
\vspace{-1cm}
\begin{equation}\leqnomode
    \tag{b} \hspace{-9.25cm} Λ → p + e^{-} + \bar{ν}_e
  \end{equation}
  where the quarks composition of the $D^{-}$ and the $Λ$ are given in Table 3.3. 
\end{mdframed}
Quark composition:
\begin{align}
    &D^{-} = d \bar{c} \\
    &K^{0} = d \bar{s} \\
    &π^{-} = d \bar{u} \\
    &Λ = uds \\
    &p = uud 
\end{align}
\subsection*{a)}
\begin{equation}
  D^{-} → K^{0} + π^{-} \quad =  \quad   d \bar{c} → d \bar{s} + d \bar{u}
\end{equation}


\section*{Problem 2 (3.13)}
\begin{mdframed}
    The particle $Y^{-}$ can be produced in the strong interaction process $K^{-} + p → K^{+} + Y^{-}$. Deduce its baryon number, strangeness, charm, and bottom, and, using these, its quark content. The $Y^{-}$(1311) decays by the reaction $Y^{-} → Λ + π^{-}$. Give a rough estimate of its lifetime.
\end{mdframed}

\section*{Problem 3 (3.11)}
\begin{mdframed}
    Find the values of the parity $P$ and, where appropriate, the charge conjugation $C$ for the ground-state ($J$ = 0) mesons $π^{±}$ and $π^{0}$, and their first
    excited ($J$ = 1) states $ρ^{±}$ and $ρ^{0}$, where the latter have a mass of about
    770 MeV/$c^2$. Why does the charged pion have a longer lifetime than the
    charged $ρ$? Explain also why the decay $ρ^{0}$ $→ π^{+} π^{-}$ has been observed, but
    not the decay $ρ^{0} →  π^{0} π^{0}$
\end{mdframed}

\section*{Problem 4 (3.21)}
\begin{mdframed}
    Derive the allowed combinations of charm $C$ and electric charge $Q$ for mesons and baryons in the simple quark model. 
\end{mdframed}




\end{document}