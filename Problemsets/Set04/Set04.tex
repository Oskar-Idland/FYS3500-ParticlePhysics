\documentclass{article}
\usepackage{amsmath}
\usepackage{titlesec}
% \usepackage[mathletters]{ucs}
\usepackage{mathtools} %for \abs{x}
\usepackage[warnings-off={mathtools-colon,mathtools-overbracket}]{unicode-math}
% \setmainfont{TeX Gyre Schola}
% \setmathfont{TeX Gyre Schola Math}
% \usepackage[utf8x]{inputenc}
\usepackage{fontenc}
\usepackage[margin=1.5in]{geometry}
\usepackage{enumerate}
\newtheorem{theorem}{Theorem}
\usepackage[dvipsnames]{xcolor}
\usepackage{pgfplots}
\pgfplotsset{compat=1.18}
\setlength{\parindent}{0cm}
\usepackage{graphics}
\usepackage{graphicx} % Required for including images
\usepackage{subcaption}
\usepackage{bigintcalc}
\usepackage{pythonhighlight} %for pythonkode \begin{python}   \end{python}
\usepackage{appendix}
\usepackage{arydshln}
\usepackage{physics}
\usepackage{booktabs} 
\usepackage{adjustbox}
\usepackage{mdframed}
\usepackage{relsize}
\usepackage{physics}
\usepackage[thinc]{esdiff}
\usepackage{esint}  %for lukket-linje-integral
\usepackage{xfrac} %for sfrac
\usepackage[colorlinks=true,linktoc=page]{hyperref} %for linker, må ha med hypersetup
\usepackage[noabbrev, nameinlink]{cleveref} % to be loaded after hyperref
% \usepackage{amssymb} %\mathbb{R} for reelle tall, \mathcal{B} for "matte"-font
\usepackage{listings} %for kode/lstlisting
\usepackage{verbatim}
\usepackage{graphicx,wrapfig,lipsum,caption} %for wrapping av bilder
\usepackage[english]{babel}
\usepackage{cancel}
% \usepackage{alphabeta}
\usepackage{mhchem} % for atom notasjon
% \definecolor{codegreen}{rgb}{0,0.6,0}
% \definecolor{codegray}{rgb}{0.5,0.5,0.5}
% \definecolor{codepurple}{rgb}{0.58,0,0.82}
% \definecolor{backcolour}{rgb}{0.95,0.95,0.92}
% \lstdefinestyle{mystyle}{
%     backgroundcolor=\color{backcolour},   
%     commentstyle=\color{codegreen},
%     keywordstyle=\color{magenta},
%     numberstyle=\tiny\color{codegray},
%     stringstyle=\color{codepurple},
%     basicstyle=\ttfamily\footnotesize,
%     breakatwhitespace=false,         
%     breaklines=true,                 
%     captionpos=b,                    
%     keepspaces=true,                 
%     numbers=left,                    
%     numbersep=5pt,                  
%     showspaces=false,                
%     showstringspaces=false,
%     showtabs=false,                  
%     tabsize=2
% }

% \lstset{style=mystyle}

\author{Oskar Idland}
\title{Problem Set 4}
\date{}
\begin{document}
% \tableofcontents
\maketitle
\newpage

\subsection*{Problem 1}
\subsection*{a)}
\begin{itemize}
    \item The standard model attempts to describe the universe through particles which interacts by exchanging force-carrying particles.
    \item All particles have intrinsic properties such as mass, charge, spin, color, chirality and so on. Force-carrying particles interact with one or more of these properties, allowing them to change from one particle, to another. 
    \item We have different groups of particles. 
    \begin{itemize}
        \item \textbf{Fermions:} Either quarks or leptons with half-integer spin and different charge, color and mass. These are divided in three generations, increasing in mass for each. 
        \item \textbf{Bosons:} Particles with integer spin, like the force carrying particles with different mass, color, and charge. Composite bosons comes from adding fermions together (like mesons or other hadrons)
        \item \textbf{Hadrons:} Two (mesons), or more quarks held together by the strong force. If greater than one, and odd, it is called a baryon. 
    \end{itemize}
\end{itemize}

\subsection*{b)}
The standard model does not describe gravity. 
\begin{enumerate}
    \item \textbf{Strong nuclear force:} Felt by particles with color, like quarks. 
    \item \textbf{Weak nuclear force:} Felt by left-chiral fermions. 
    \item \textbf{Electromagnetism:} Felt by all fermions with charge, like the quarks, electron, muon and tau. 
\end{enumerate}

\subsection*{c)}
Interactions between particles goes through force-carrying particles. 
\begin{itemize}
    \item \textbf{Strong nuclear force:} Gluon
    \item \textbf{Weak nuclear force:} The $Z^{0}$, $W^{+}$ and $W^{-}$-boson. 
    \item \textbf{Electromagnetism:} The photon $γ$
\end{itemize}

\section*{Problem 2}
\subsection*{a)}
Feynman diagrams depict particle interaction through lines, representing different types of particles, and vertices showing where they interact. By convention, time flows from left to right. 

\subsection*{b)}
\begin{itemize}
    \item Spin-half fermions are whole lines with an arrow. Arrows pointing forwards are fermions, and backwards are anti-fermions. 
    \item Gluons are drawn like springs.
    \item Spin-1 bosons are drawn as waves.
    \item Spin-0 boson are drawn with dashed lines. 
    \item Fermion lines must always be connected through the entire diagram. 
\end{itemize}

\subsection*{c)}
Virtual particles are identified with an asterix (*) and have vertices on each side. 

\subsection*{d)}
The order of a Feynman diagram is how many chains of vertices are present. Higher-order diagrams are less probable as they require multiple interactions. 

\subsection*{e)}
The lepton, anti-lepton pairs have arrows in the wrong directions. 


\section*{Problem 3}
\subsection*{a)}
In this diagram, an electron and anti-electron collide, create a photon, which then becomes an electron and anti-electron. 

\subsection*{b)}
In this diagram, two electrons interact by exchanging a photon. Which is giving or receiving, is not specified. 

\subsection*{c)}
In this diagram, an electron emits a photon, then absorbs one. It could also absorb, before emitting, as the figure does not specify the time order. 




\end{document}